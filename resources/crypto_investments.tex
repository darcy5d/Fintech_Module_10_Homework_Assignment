\documentclass[11pt]{article}

    \usepackage[breakable]{tcolorbox}
    \usepackage{parskip} % Stop auto-indenting (to mimic markdown behaviour)
    

    % Basic figure setup, for now with no caption control since it's done
    % automatically by Pandoc (which extracts ![](path) syntax from Markdown).
    \usepackage{graphicx}
    % Maintain compatibility with old templates. Remove in nbconvert 6.0
    \let\Oldincludegraphics\includegraphics
    % Ensure that by default, figures have no caption (until we provide a
    % proper Figure object with a Caption API and a way to capture that
    % in the conversion process - todo).
    \usepackage{caption}
    \DeclareCaptionFormat{nocaption}{}
    \captionsetup{format=nocaption,aboveskip=0pt,belowskip=0pt}

    \usepackage{float}
    \floatplacement{figure}{H} % forces figures to be placed at the correct location
    \usepackage{xcolor} % Allow colors to be defined
    \usepackage{enumerate} % Needed for markdown enumerations to work
    \usepackage{geometry} % Used to adjust the document margins
    \usepackage{amsmath} % Equations
    \usepackage{amssymb} % Equations
    \usepackage{textcomp} % defines textquotesingle
    % Hack from http://tex.stackexchange.com/a/47451/13684:
    \AtBeginDocument{%
        \def\PYZsq{\textquotesingle}% Upright quotes in Pygmentized code
    }
    \usepackage{upquote} % Upright quotes for verbatim code
    \usepackage{eurosym} % defines \euro

    \usepackage{iftex}
    \ifPDFTeX
        \usepackage[T1]{fontenc}
        \IfFileExists{alphabeta.sty}{
              \usepackage{alphabeta}
          }{
              \usepackage[mathletters]{ucs}
              \usepackage[utf8x]{inputenc}
          }
    \else
        \usepackage{fontspec}
        \usepackage{unicode-math}
    \fi

    \usepackage{fancyvrb} % verbatim replacement that allows latex
    \usepackage{grffile} % extends the file name processing of package graphics
                         % to support a larger range
    \makeatletter % fix for old versions of grffile with XeLaTeX
    \@ifpackagelater{grffile}{2019/11/01}
    {
      % Do nothing on new versions
    }
    {
      \def\Gread@@xetex#1{%
        \IfFileExists{"\Gin@base".bb}%
        {\Gread@eps{\Gin@base.bb}}%
        {\Gread@@xetex@aux#1}%
      }
    }
    \makeatother
    \usepackage[Export]{adjustbox} % Used to constrain images to a maximum size
    \adjustboxset{max size={0.9\linewidth}{0.9\paperheight}}

    % The hyperref package gives us a pdf with properly built
    % internal navigation ('pdf bookmarks' for the table of contents,
    % internal cross-reference links, web links for URLs, etc.)
    \usepackage{hyperref}
    % The default LaTeX title has an obnoxious amount of whitespace. By default,
    % titling removes some of it. It also provides customization options.
    \usepackage{titling}
    \usepackage{longtable} % longtable support required by pandoc >1.10
    \usepackage{booktabs}  % table support for pandoc > 1.12.2
    \usepackage{array}     % table support for pandoc >= 2.11.3
    \usepackage{calc}      % table minipage width calculation for pandoc >= 2.11.1
    \usepackage[inline]{enumitem} % IRkernel/repr support (it uses the enumerate* environment)
    \usepackage[normalem]{ulem} % ulem is needed to support strikethroughs (\sout)
                                % normalem makes italics be italics, not underlines
    \usepackage{soul}      % strikethrough (\st) support for pandoc >= 3.0.0
    \usepackage{mathrsfs}
    

    
    % Colors for the hyperref package
    \definecolor{urlcolor}{rgb}{0,.145,.698}
    \definecolor{linkcolor}{rgb}{.71,0.21,0.01}
    \definecolor{citecolor}{rgb}{.12,.54,.11}

    % ANSI colors
    \definecolor{ansi-black}{HTML}{3E424D}
    \definecolor{ansi-black-intense}{HTML}{282C36}
    \definecolor{ansi-red}{HTML}{E75C58}
    \definecolor{ansi-red-intense}{HTML}{B22B31}
    \definecolor{ansi-green}{HTML}{00A250}
    \definecolor{ansi-green-intense}{HTML}{007427}
    \definecolor{ansi-yellow}{HTML}{DDB62B}
    \definecolor{ansi-yellow-intense}{HTML}{B27D12}
    \definecolor{ansi-blue}{HTML}{208FFB}
    \definecolor{ansi-blue-intense}{HTML}{0065CA}
    \definecolor{ansi-magenta}{HTML}{D160C4}
    \definecolor{ansi-magenta-intense}{HTML}{A03196}
    \definecolor{ansi-cyan}{HTML}{60C6C8}
    \definecolor{ansi-cyan-intense}{HTML}{258F8F}
    \definecolor{ansi-white}{HTML}{C5C1B4}
    \definecolor{ansi-white-intense}{HTML}{A1A6B2}
    \definecolor{ansi-default-inverse-fg}{HTML}{FFFFFF}
    \definecolor{ansi-default-inverse-bg}{HTML}{000000}

    % common color for the border for error outputs.
    \definecolor{outerrorbackground}{HTML}{FFDFDF}

    % commands and environments needed by pandoc snippets
    % extracted from the output of `pandoc -s`
    \providecommand{\tightlist}{%
      \setlength{\itemsep}{0pt}\setlength{\parskip}{0pt}}
    \DefineVerbatimEnvironment{Highlighting}{Verbatim}{commandchars=\\\{\}}
    % Add ',fontsize=\small' for more characters per line
    \newenvironment{Shaded}{}{}
    \newcommand{\KeywordTok}[1]{\textcolor[rgb]{0.00,0.44,0.13}{\textbf{{#1}}}}
    \newcommand{\DataTypeTok}[1]{\textcolor[rgb]{0.56,0.13,0.00}{{#1}}}
    \newcommand{\DecValTok}[1]{\textcolor[rgb]{0.25,0.63,0.44}{{#1}}}
    \newcommand{\BaseNTok}[1]{\textcolor[rgb]{0.25,0.63,0.44}{{#1}}}
    \newcommand{\FloatTok}[1]{\textcolor[rgb]{0.25,0.63,0.44}{{#1}}}
    \newcommand{\CharTok}[1]{\textcolor[rgb]{0.25,0.44,0.63}{{#1}}}
    \newcommand{\StringTok}[1]{\textcolor[rgb]{0.25,0.44,0.63}{{#1}}}
    \newcommand{\CommentTok}[1]{\textcolor[rgb]{0.38,0.63,0.69}{\textit{{#1}}}}
    \newcommand{\OtherTok}[1]{\textcolor[rgb]{0.00,0.44,0.13}{{#1}}}
    \newcommand{\AlertTok}[1]{\textcolor[rgb]{1.00,0.00,0.00}{\textbf{{#1}}}}
    \newcommand{\FunctionTok}[1]{\textcolor[rgb]{0.02,0.16,0.49}{{#1}}}
    \newcommand{\RegionMarkerTok}[1]{{#1}}
    \newcommand{\ErrorTok}[1]{\textcolor[rgb]{1.00,0.00,0.00}{\textbf{{#1}}}}
    \newcommand{\NormalTok}[1]{{#1}}

    % Additional commands for more recent versions of Pandoc
    \newcommand{\ConstantTok}[1]{\textcolor[rgb]{0.53,0.00,0.00}{{#1}}}
    \newcommand{\SpecialCharTok}[1]{\textcolor[rgb]{0.25,0.44,0.63}{{#1}}}
    \newcommand{\VerbatimStringTok}[1]{\textcolor[rgb]{0.25,0.44,0.63}{{#1}}}
    \newcommand{\SpecialStringTok}[1]{\textcolor[rgb]{0.73,0.40,0.53}{{#1}}}
    \newcommand{\ImportTok}[1]{{#1}}
    \newcommand{\DocumentationTok}[1]{\textcolor[rgb]{0.73,0.13,0.13}{\textit{{#1}}}}
    \newcommand{\AnnotationTok}[1]{\textcolor[rgb]{0.38,0.63,0.69}{\textbf{\textit{{#1}}}}}
    \newcommand{\CommentVarTok}[1]{\textcolor[rgb]{0.38,0.63,0.69}{\textbf{\textit{{#1}}}}}
    \newcommand{\VariableTok}[1]{\textcolor[rgb]{0.10,0.09,0.49}{{#1}}}
    \newcommand{\ControlFlowTok}[1]{\textcolor[rgb]{0.00,0.44,0.13}{\textbf{{#1}}}}
    \newcommand{\OperatorTok}[1]{\textcolor[rgb]{0.40,0.40,0.40}{{#1}}}
    \newcommand{\BuiltInTok}[1]{{#1}}
    \newcommand{\ExtensionTok}[1]{{#1}}
    \newcommand{\PreprocessorTok}[1]{\textcolor[rgb]{0.74,0.48,0.00}{{#1}}}
    \newcommand{\AttributeTok}[1]{\textcolor[rgb]{0.49,0.56,0.16}{{#1}}}
    \newcommand{\InformationTok}[1]{\textcolor[rgb]{0.38,0.63,0.69}{\textbf{\textit{{#1}}}}}
    \newcommand{\WarningTok}[1]{\textcolor[rgb]{0.38,0.63,0.69}{\textbf{\textit{{#1}}}}}


    % Define a nice break command that doesn't care if a line doesn't already
    % exist.
    \def\br{\hspace*{\fill} \\* }
    % Math Jax compatibility definitions
    \def\gt{>}
    \def\lt{<}
    \let\Oldtex\TeX
    \let\Oldlatex\LaTeX
    \renewcommand{\TeX}{\textrm{\Oldtex}}
    \renewcommand{\LaTeX}{\textrm{\Oldlatex}}
    % Document parameters
    % Document title
    \title{crypto\_investments}
    
    
    
    
    
    
    
% Pygments definitions
\makeatletter
\def\PY@reset{\let\PY@it=\relax \let\PY@bf=\relax%
    \let\PY@ul=\relax \let\PY@tc=\relax%
    \let\PY@bc=\relax \let\PY@ff=\relax}
\def\PY@tok#1{\csname PY@tok@#1\endcsname}
\def\PY@toks#1+{\ifx\relax#1\empty\else%
    \PY@tok{#1}\expandafter\PY@toks\fi}
\def\PY@do#1{\PY@bc{\PY@tc{\PY@ul{%
    \PY@it{\PY@bf{\PY@ff{#1}}}}}}}
\def\PY#1#2{\PY@reset\PY@toks#1+\relax+\PY@do{#2}}

\@namedef{PY@tok@w}{\def\PY@tc##1{\textcolor[rgb]{0.73,0.73,0.73}{##1}}}
\@namedef{PY@tok@c}{\let\PY@it=\textit\def\PY@tc##1{\textcolor[rgb]{0.24,0.48,0.48}{##1}}}
\@namedef{PY@tok@cp}{\def\PY@tc##1{\textcolor[rgb]{0.61,0.40,0.00}{##1}}}
\@namedef{PY@tok@k}{\let\PY@bf=\textbf\def\PY@tc##1{\textcolor[rgb]{0.00,0.50,0.00}{##1}}}
\@namedef{PY@tok@kp}{\def\PY@tc##1{\textcolor[rgb]{0.00,0.50,0.00}{##1}}}
\@namedef{PY@tok@kt}{\def\PY@tc##1{\textcolor[rgb]{0.69,0.00,0.25}{##1}}}
\@namedef{PY@tok@o}{\def\PY@tc##1{\textcolor[rgb]{0.40,0.40,0.40}{##1}}}
\@namedef{PY@tok@ow}{\let\PY@bf=\textbf\def\PY@tc##1{\textcolor[rgb]{0.67,0.13,1.00}{##1}}}
\@namedef{PY@tok@nb}{\def\PY@tc##1{\textcolor[rgb]{0.00,0.50,0.00}{##1}}}
\@namedef{PY@tok@nf}{\def\PY@tc##1{\textcolor[rgb]{0.00,0.00,1.00}{##1}}}
\@namedef{PY@tok@nc}{\let\PY@bf=\textbf\def\PY@tc##1{\textcolor[rgb]{0.00,0.00,1.00}{##1}}}
\@namedef{PY@tok@nn}{\let\PY@bf=\textbf\def\PY@tc##1{\textcolor[rgb]{0.00,0.00,1.00}{##1}}}
\@namedef{PY@tok@ne}{\let\PY@bf=\textbf\def\PY@tc##1{\textcolor[rgb]{0.80,0.25,0.22}{##1}}}
\@namedef{PY@tok@nv}{\def\PY@tc##1{\textcolor[rgb]{0.10,0.09,0.49}{##1}}}
\@namedef{PY@tok@no}{\def\PY@tc##1{\textcolor[rgb]{0.53,0.00,0.00}{##1}}}
\@namedef{PY@tok@nl}{\def\PY@tc##1{\textcolor[rgb]{0.46,0.46,0.00}{##1}}}
\@namedef{PY@tok@ni}{\let\PY@bf=\textbf\def\PY@tc##1{\textcolor[rgb]{0.44,0.44,0.44}{##1}}}
\@namedef{PY@tok@na}{\def\PY@tc##1{\textcolor[rgb]{0.41,0.47,0.13}{##1}}}
\@namedef{PY@tok@nt}{\let\PY@bf=\textbf\def\PY@tc##1{\textcolor[rgb]{0.00,0.50,0.00}{##1}}}
\@namedef{PY@tok@nd}{\def\PY@tc##1{\textcolor[rgb]{0.67,0.13,1.00}{##1}}}
\@namedef{PY@tok@s}{\def\PY@tc##1{\textcolor[rgb]{0.73,0.13,0.13}{##1}}}
\@namedef{PY@tok@sd}{\let\PY@it=\textit\def\PY@tc##1{\textcolor[rgb]{0.73,0.13,0.13}{##1}}}
\@namedef{PY@tok@si}{\let\PY@bf=\textbf\def\PY@tc##1{\textcolor[rgb]{0.64,0.35,0.47}{##1}}}
\@namedef{PY@tok@se}{\let\PY@bf=\textbf\def\PY@tc##1{\textcolor[rgb]{0.67,0.36,0.12}{##1}}}
\@namedef{PY@tok@sr}{\def\PY@tc##1{\textcolor[rgb]{0.64,0.35,0.47}{##1}}}
\@namedef{PY@tok@ss}{\def\PY@tc##1{\textcolor[rgb]{0.10,0.09,0.49}{##1}}}
\@namedef{PY@tok@sx}{\def\PY@tc##1{\textcolor[rgb]{0.00,0.50,0.00}{##1}}}
\@namedef{PY@tok@m}{\def\PY@tc##1{\textcolor[rgb]{0.40,0.40,0.40}{##1}}}
\@namedef{PY@tok@gh}{\let\PY@bf=\textbf\def\PY@tc##1{\textcolor[rgb]{0.00,0.00,0.50}{##1}}}
\@namedef{PY@tok@gu}{\let\PY@bf=\textbf\def\PY@tc##1{\textcolor[rgb]{0.50,0.00,0.50}{##1}}}
\@namedef{PY@tok@gd}{\def\PY@tc##1{\textcolor[rgb]{0.63,0.00,0.00}{##1}}}
\@namedef{PY@tok@gi}{\def\PY@tc##1{\textcolor[rgb]{0.00,0.52,0.00}{##1}}}
\@namedef{PY@tok@gr}{\def\PY@tc##1{\textcolor[rgb]{0.89,0.00,0.00}{##1}}}
\@namedef{PY@tok@ge}{\let\PY@it=\textit}
\@namedef{PY@tok@gs}{\let\PY@bf=\textbf}
\@namedef{PY@tok@ges}{\let\PY@bf=\textbf\let\PY@it=\textit}
\@namedef{PY@tok@gp}{\let\PY@bf=\textbf\def\PY@tc##1{\textcolor[rgb]{0.00,0.00,0.50}{##1}}}
\@namedef{PY@tok@go}{\def\PY@tc##1{\textcolor[rgb]{0.44,0.44,0.44}{##1}}}
\@namedef{PY@tok@gt}{\def\PY@tc##1{\textcolor[rgb]{0.00,0.27,0.87}{##1}}}
\@namedef{PY@tok@err}{\def\PY@bc##1{{\setlength{\fboxsep}{\string -\fboxrule}\fcolorbox[rgb]{1.00,0.00,0.00}{1,1,1}{\strut ##1}}}}
\@namedef{PY@tok@kc}{\let\PY@bf=\textbf\def\PY@tc##1{\textcolor[rgb]{0.00,0.50,0.00}{##1}}}
\@namedef{PY@tok@kd}{\let\PY@bf=\textbf\def\PY@tc##1{\textcolor[rgb]{0.00,0.50,0.00}{##1}}}
\@namedef{PY@tok@kn}{\let\PY@bf=\textbf\def\PY@tc##1{\textcolor[rgb]{0.00,0.50,0.00}{##1}}}
\@namedef{PY@tok@kr}{\let\PY@bf=\textbf\def\PY@tc##1{\textcolor[rgb]{0.00,0.50,0.00}{##1}}}
\@namedef{PY@tok@bp}{\def\PY@tc##1{\textcolor[rgb]{0.00,0.50,0.00}{##1}}}
\@namedef{PY@tok@fm}{\def\PY@tc##1{\textcolor[rgb]{0.00,0.00,1.00}{##1}}}
\@namedef{PY@tok@vc}{\def\PY@tc##1{\textcolor[rgb]{0.10,0.09,0.49}{##1}}}
\@namedef{PY@tok@vg}{\def\PY@tc##1{\textcolor[rgb]{0.10,0.09,0.49}{##1}}}
\@namedef{PY@tok@vi}{\def\PY@tc##1{\textcolor[rgb]{0.10,0.09,0.49}{##1}}}
\@namedef{PY@tok@vm}{\def\PY@tc##1{\textcolor[rgb]{0.10,0.09,0.49}{##1}}}
\@namedef{PY@tok@sa}{\def\PY@tc##1{\textcolor[rgb]{0.73,0.13,0.13}{##1}}}
\@namedef{PY@tok@sb}{\def\PY@tc##1{\textcolor[rgb]{0.73,0.13,0.13}{##1}}}
\@namedef{PY@tok@sc}{\def\PY@tc##1{\textcolor[rgb]{0.73,0.13,0.13}{##1}}}
\@namedef{PY@tok@dl}{\def\PY@tc##1{\textcolor[rgb]{0.73,0.13,0.13}{##1}}}
\@namedef{PY@tok@s2}{\def\PY@tc##1{\textcolor[rgb]{0.73,0.13,0.13}{##1}}}
\@namedef{PY@tok@sh}{\def\PY@tc##1{\textcolor[rgb]{0.73,0.13,0.13}{##1}}}
\@namedef{PY@tok@s1}{\def\PY@tc##1{\textcolor[rgb]{0.73,0.13,0.13}{##1}}}
\@namedef{PY@tok@mb}{\def\PY@tc##1{\textcolor[rgb]{0.40,0.40,0.40}{##1}}}
\@namedef{PY@tok@mf}{\def\PY@tc##1{\textcolor[rgb]{0.40,0.40,0.40}{##1}}}
\@namedef{PY@tok@mh}{\def\PY@tc##1{\textcolor[rgb]{0.40,0.40,0.40}{##1}}}
\@namedef{PY@tok@mi}{\def\PY@tc##1{\textcolor[rgb]{0.40,0.40,0.40}{##1}}}
\@namedef{PY@tok@il}{\def\PY@tc##1{\textcolor[rgb]{0.40,0.40,0.40}{##1}}}
\@namedef{PY@tok@mo}{\def\PY@tc##1{\textcolor[rgb]{0.40,0.40,0.40}{##1}}}
\@namedef{PY@tok@ch}{\let\PY@it=\textit\def\PY@tc##1{\textcolor[rgb]{0.24,0.48,0.48}{##1}}}
\@namedef{PY@tok@cm}{\let\PY@it=\textit\def\PY@tc##1{\textcolor[rgb]{0.24,0.48,0.48}{##1}}}
\@namedef{PY@tok@cpf}{\let\PY@it=\textit\def\PY@tc##1{\textcolor[rgb]{0.24,0.48,0.48}{##1}}}
\@namedef{PY@tok@c1}{\let\PY@it=\textit\def\PY@tc##1{\textcolor[rgb]{0.24,0.48,0.48}{##1}}}
\@namedef{PY@tok@cs}{\let\PY@it=\textit\def\PY@tc##1{\textcolor[rgb]{0.24,0.48,0.48}{##1}}}

\def\PYZbs{\char`\\}
\def\PYZus{\char`\_}
\def\PYZob{\char`\{}
\def\PYZcb{\char`\}}
\def\PYZca{\char`\^}
\def\PYZam{\char`\&}
\def\PYZlt{\char`\<}
\def\PYZgt{\char`\>}
\def\PYZsh{\char`\#}
\def\PYZpc{\char`\%}
\def\PYZdl{\char`\$}
\def\PYZhy{\char`\-}
\def\PYZsq{\char`\'}
\def\PYZdq{\char`\"}
\def\PYZti{\char`\~}
% for compatibility with earlier versions
\def\PYZat{@}
\def\PYZlb{[}
\def\PYZrb{]}
\makeatother


    % For linebreaks inside Verbatim environment from package fancyvrb.
    \makeatletter
        \newbox\Wrappedcontinuationbox
        \newbox\Wrappedvisiblespacebox
        \newcommand*\Wrappedvisiblespace {\textcolor{red}{\textvisiblespace}}
        \newcommand*\Wrappedcontinuationsymbol {\textcolor{red}{\llap{\tiny$\m@th\hookrightarrow$}}}
        \newcommand*\Wrappedcontinuationindent {3ex }
        \newcommand*\Wrappedafterbreak {\kern\Wrappedcontinuationindent\copy\Wrappedcontinuationbox}
        % Take advantage of the already applied Pygments mark-up to insert
        % potential linebreaks for TeX processing.
        %        {, <, #, %, $, ' and ": go to next line.
        %        _, }, ^, &, >, - and ~: stay at end of broken line.
        % Use of \textquotesingle for straight quote.
        \newcommand*\Wrappedbreaksatspecials {%
            \def\PYGZus{\discretionary{\char`\_}{\Wrappedafterbreak}{\char`\_}}%
            \def\PYGZob{\discretionary{}{\Wrappedafterbreak\char`\{}{\char`\{}}%
            \def\PYGZcb{\discretionary{\char`\}}{\Wrappedafterbreak}{\char`\}}}%
            \def\PYGZca{\discretionary{\char`\^}{\Wrappedafterbreak}{\char`\^}}%
            \def\PYGZam{\discretionary{\char`\&}{\Wrappedafterbreak}{\char`\&}}%
            \def\PYGZlt{\discretionary{}{\Wrappedafterbreak\char`\<}{\char`\<}}%
            \def\PYGZgt{\discretionary{\char`\>}{\Wrappedafterbreak}{\char`\>}}%
            \def\PYGZsh{\discretionary{}{\Wrappedafterbreak\char`\#}{\char`\#}}%
            \def\PYGZpc{\discretionary{}{\Wrappedafterbreak\char`\%}{\char`\%}}%
            \def\PYGZdl{\discretionary{}{\Wrappedafterbreak\char`\$}{\char`\$}}%
            \def\PYGZhy{\discretionary{\char`\-}{\Wrappedafterbreak}{\char`\-}}%
            \def\PYGZsq{\discretionary{}{\Wrappedafterbreak\textquotesingle}{\textquotesingle}}%
            \def\PYGZdq{\discretionary{}{\Wrappedafterbreak\char`\"}{\char`\"}}%
            \def\PYGZti{\discretionary{\char`\~}{\Wrappedafterbreak}{\char`\~}}%
        }
        % Some characters . , ; ? ! / are not pygmentized.
        % This macro makes them "active" and they will insert potential linebreaks
        \newcommand*\Wrappedbreaksatpunct {%
            \lccode`\~`\.\lowercase{\def~}{\discretionary{\hbox{\char`\.}}{\Wrappedafterbreak}{\hbox{\char`\.}}}%
            \lccode`\~`\,\lowercase{\def~}{\discretionary{\hbox{\char`\,}}{\Wrappedafterbreak}{\hbox{\char`\,}}}%
            \lccode`\~`\;\lowercase{\def~}{\discretionary{\hbox{\char`\;}}{\Wrappedafterbreak}{\hbox{\char`\;}}}%
            \lccode`\~`\:\lowercase{\def~}{\discretionary{\hbox{\char`\:}}{\Wrappedafterbreak}{\hbox{\char`\:}}}%
            \lccode`\~`\?\lowercase{\def~}{\discretionary{\hbox{\char`\?}}{\Wrappedafterbreak}{\hbox{\char`\?}}}%
            \lccode`\~`\!\lowercase{\def~}{\discretionary{\hbox{\char`\!}}{\Wrappedafterbreak}{\hbox{\char`\!}}}%
            \lccode`\~`\/\lowercase{\def~}{\discretionary{\hbox{\char`\/}}{\Wrappedafterbreak}{\hbox{\char`\/}}}%
            \catcode`\.\active
            \catcode`\,\active
            \catcode`\;\active
            \catcode`\:\active
            \catcode`\?\active
            \catcode`\!\active
            \catcode`\/\active
            \lccode`\~`\~
        }
    \makeatother

    \let\OriginalVerbatim=\Verbatim
    \makeatletter
    \renewcommand{\Verbatim}[1][1]{%
        %\parskip\z@skip
        \sbox\Wrappedcontinuationbox {\Wrappedcontinuationsymbol}%
        \sbox\Wrappedvisiblespacebox {\FV@SetupFont\Wrappedvisiblespace}%
        \def\FancyVerbFormatLine ##1{\hsize\linewidth
            \vtop{\raggedright\hyphenpenalty\z@\exhyphenpenalty\z@
                \doublehyphendemerits\z@\finalhyphendemerits\z@
                \strut ##1\strut}%
        }%
        % If the linebreak is at a space, the latter will be displayed as visible
        % space at end of first line, and a continuation symbol starts next line.
        % Stretch/shrink are however usually zero for typewriter font.
        \def\FV@Space {%
            \nobreak\hskip\z@ plus\fontdimen3\font minus\fontdimen4\font
            \discretionary{\copy\Wrappedvisiblespacebox}{\Wrappedafterbreak}
            {\kern\fontdimen2\font}%
        }%

        % Allow breaks at special characters using \PYG... macros.
        \Wrappedbreaksatspecials
        % Breaks at punctuation characters . , ; ? ! and / need catcode=\active
        \OriginalVerbatim[#1,codes*=\Wrappedbreaksatpunct]%
    }
    \makeatother

    % Exact colors from NB
    \definecolor{incolor}{HTML}{303F9F}
    \definecolor{outcolor}{HTML}{D84315}
    \definecolor{cellborder}{HTML}{CFCFCF}
    \definecolor{cellbackground}{HTML}{F7F7F7}

    % prompt
    \makeatletter
    \newcommand{\boxspacing}{\kern\kvtcb@left@rule\kern\kvtcb@boxsep}
    \makeatother
    \newcommand{\prompt}[4]{
        {\ttfamily\llap{{\color{#2}[#3]:\hspace{3pt}#4}}\vspace{-\baselineskip}}
    }
    

    
    % Prevent overflowing lines due to hard-to-break entities
    \sloppy
    % Setup hyperref package
    \hypersetup{
      breaklinks=true,  % so long urls are correctly broken across lines
      colorlinks=true,
      urlcolor=urlcolor,
      linkcolor=linkcolor,
      citecolor=citecolor,
      }
    % Slightly bigger margins than the latex defaults
    
    \geometry{verbose,tmargin=1in,bmargin=1in,lmargin=1in,rmargin=1in}
    
    

\begin{document}
    
    \maketitle
    
    

    
    \hypertarget{module-10-application}{%
\section{Module 10 Application}\label{module-10-application}}

\hypertarget{challenge-crypto-clustering}{%
\subsection{Challenge: Crypto
Clustering}\label{challenge-crypto-clustering}}

In this Challenge, you'll combine your financial Python programming
skills with the new unsupervised learning skills that you acquired in
this module.

The CSV file provided for this challenge contains price change data of
cryptocurrencies in different periods.

The steps for this challenge are broken out into the following sections:

\begin{itemize}
\tightlist
\item
  Import the Data (provided in the starter code)
\item
  Prepare the Data (provided in the starter code)
\item
  Find the Best Value for \texttt{k} Using the Original Data
\item
  Cluster Cryptocurrencies with K-means Using the Original Data
\item
  Optimize Clusters with Principal Component Analysis
\item
  Find the Best Value for \texttt{k} Using the PCA Data
\item
  Cluster the Cryptocurrencies with K-means Using the PCA Data
\item
  Visualize and Compare the Results
\end{itemize}

    \hypertarget{import-the-data}{%
\subsubsection{Import the Data}\label{import-the-data}}

This section imports the data into a new DataFrame. It follows these
steps:

\begin{enumerate}
\def\labelenumi{\arabic{enumi}.}
\item
  Read the ``crypto\_market\_data.csv'' file from the Resources folder
  into a DataFrame, and use \texttt{index\_col="coin\_id"} to set the
  cryptocurrency name as the index. Review the DataFrame.
\item
  Generate the summary statistics, and use HvPlot to visualize your data
  to observe what your DataFrame contains.
\end{enumerate}

\begin{quote}
\textbf{Rewind:} The
\href{https://pandas.pydata.org/pandas-docs/stable/reference/api/pandas.DataFrame.describe.html}{Pandas\texttt{describe()}function}
generates summary statistics for a DataFrame.
\end{quote}

    \begin{tcolorbox}[breakable, size=fbox, boxrule=1pt, pad at break*=1mm,colback=cellbackground, colframe=cellborder]
\prompt{In}{incolor}{ }{\boxspacing}
\begin{Verbatim}[commandchars=\\\{\}]
\PY{c+c1}{\PYZsh{} Import required libraries and dependencies}
\PY{k+kn}{import} \PY{n+nn}{pandas} \PY{k}{as} \PY{n+nn}{pd}
\PY{k+kn}{import} \PY{n+nn}{hvplot}\PY{n+nn}{.}\PY{n+nn}{pandas}
\PY{k+kn}{from} \PY{n+nn}{pathlib} \PY{k+kn}{import} \PY{n}{Path}
\PY{k+kn}{from} \PY{n+nn}{sklearn}\PY{n+nn}{.}\PY{n+nn}{cluster} \PY{k+kn}{import} \PY{n}{KMeans}
\PY{k+kn}{from} \PY{n+nn}{sklearn}\PY{n+nn}{.}\PY{n+nn}{decomposition} \PY{k+kn}{import} \PY{n}{PCA}
\PY{k+kn}{from} \PY{n+nn}{sklearn}\PY{n+nn}{.}\PY{n+nn}{preprocessing} \PY{k+kn}{import} \PY{n}{StandardScaler}
\end{Verbatim}
\end{tcolorbox}

    \begin{tcolorbox}[breakable, size=fbox, boxrule=1pt, pad at break*=1mm,colback=cellbackground, colframe=cellborder]
\prompt{In}{incolor}{ }{\boxspacing}
\begin{Verbatim}[commandchars=\\\{\}]
\PY{c+c1}{\PYZsh{} Load the data into a Pandas DataFrame}
\PY{n}{df\PYZus{}market\PYZus{}data} \PY{o}{=} \PY{n}{pd}\PY{o}{.}\PY{n}{read\PYZus{}csv}\PY{p}{(}
    \PY{n}{Path}\PY{p}{(}\PY{l+s+s2}{\PYZdq{}}\PY{l+s+s2}{Resources/crypto\PYZus{}market\PYZus{}data.csv}\PY{l+s+s2}{\PYZdq{}}\PY{p}{)}\PY{p}{,}
    \PY{n}{index\PYZus{}col}\PY{o}{=}\PY{l+s+s2}{\PYZdq{}}\PY{l+s+s2}{coin\PYZus{}id}\PY{l+s+s2}{\PYZdq{}}\PY{p}{)}

\PY{c+c1}{\PYZsh{} Display sample data}
\PY{n}{df\PYZus{}market\PYZus{}data}\PY{o}{.}\PY{n}{head}\PY{p}{(}\PY{l+m+mi}{10}\PY{p}{)}
\end{Verbatim}
\end{tcolorbox}

    \begin{tcolorbox}[breakable, size=fbox, boxrule=1pt, pad at break*=1mm,colback=cellbackground, colframe=cellborder]
\prompt{In}{incolor}{ }{\boxspacing}
\begin{Verbatim}[commandchars=\\\{\}]
\PY{c+c1}{\PYZsh{} Generate summary statistics}
\PY{n}{df\PYZus{}market\PYZus{}data}\PY{o}{.}\PY{n}{describe}\PY{p}{(}\PY{p}{)}
\end{Verbatim}
\end{tcolorbox}

    \begin{tcolorbox}[breakable, size=fbox, boxrule=1pt, pad at break*=1mm,colback=cellbackground, colframe=cellborder]
\prompt{In}{incolor}{ }{\boxspacing}
\begin{Verbatim}[commandchars=\\\{\}]
\PY{c+c1}{\PYZsh{} Plot your data to see what\PYZsq{}s in your DataFrame}
\PY{n}{df\PYZus{}market\PYZus{}data}\PY{o}{.}\PY{n}{hvplot}\PY{o}{.}\PY{n}{line}\PY{p}{(}
    \PY{n}{width}\PY{o}{=}\PY{l+m+mi}{800}\PY{p}{,}
    \PY{n}{height}\PY{o}{=}\PY{l+m+mi}{400}\PY{p}{,}
    \PY{n}{rot}\PY{o}{=}\PY{l+m+mi}{90}
\PY{p}{)}
\end{Verbatim}
\end{tcolorbox}

    \begin{center}\rule{0.5\linewidth}{0.5pt}\end{center}

    \hypertarget{prepare-the-data}{%
\subsubsection{Prepare the Data}\label{prepare-the-data}}

This section prepares the data before running the K-Means algorithm. It
follows these steps:

\begin{enumerate}
\def\labelenumi{\arabic{enumi}.}
\item
  Use the \texttt{StandardScaler} module from scikit-learn to normalize
  the CSV file data. This will require you to utilize the
  \texttt{fit\_transform} function.
\item
  Create a DataFrame that contains the scaled data. Be sure to set the
  \texttt{coin\_id} index from the original DataFrame as the index for
  the new DataFrame. Review the resulting DataFrame.
\end{enumerate}

    \begin{tcolorbox}[breakable, size=fbox, boxrule=1pt, pad at break*=1mm,colback=cellbackground, colframe=cellborder]
\prompt{In}{incolor}{ }{\boxspacing}
\begin{Verbatim}[commandchars=\\\{\}]
\PY{c+c1}{\PYZsh{} Use the `StandardScaler()` module from scikit\PYZhy{}learn to normalize the data from the CSV file}
\PY{n}{scaled\PYZus{}data} \PY{o}{=} \PY{n}{StandardScaler}\PY{p}{(}\PY{p}{)}\PY{o}{.}\PY{n}{fit\PYZus{}transform}\PY{p}{(}\PY{n}{df\PYZus{}market\PYZus{}data}\PY{p}{)}
\end{Verbatim}
\end{tcolorbox}

    \begin{tcolorbox}[breakable, size=fbox, boxrule=1pt, pad at break*=1mm,colback=cellbackground, colframe=cellborder]
\prompt{In}{incolor}{ }{\boxspacing}
\begin{Verbatim}[commandchars=\\\{\}]
\PY{c+c1}{\PYZsh{} Create a DataFrame with the scaled data}
\PY{n}{df\PYZus{}market\PYZus{}data\PYZus{}scaled} \PY{o}{=} \PY{n}{pd}\PY{o}{.}\PY{n}{DataFrame}\PY{p}{(}
    \PY{n}{scaled\PYZus{}data}\PY{p}{,}
    \PY{n}{columns}\PY{o}{=}\PY{n}{df\PYZus{}market\PYZus{}data}\PY{o}{.}\PY{n}{columns}
\PY{p}{)}

\PY{c+c1}{\PYZsh{} Copy the crypto names from the original data}
\PY{n}{df\PYZus{}market\PYZus{}data\PYZus{}scaled}\PY{p}{[}\PY{l+s+s2}{\PYZdq{}}\PY{l+s+s2}{coin\PYZus{}id}\PY{l+s+s2}{\PYZdq{}}\PY{p}{]} \PY{o}{=} \PY{n}{df\PYZus{}market\PYZus{}data}\PY{o}{.}\PY{n}{index}

\PY{c+c1}{\PYZsh{} Set the coinid column as index}
\PY{n}{df\PYZus{}market\PYZus{}data\PYZus{}scaled} \PY{o}{=} \PY{n}{df\PYZus{}market\PYZus{}data\PYZus{}scaled}\PY{o}{.}\PY{n}{set\PYZus{}index}\PY{p}{(}\PY{l+s+s2}{\PYZdq{}}\PY{l+s+s2}{coin\PYZus{}id}\PY{l+s+s2}{\PYZdq{}}\PY{p}{)}

\PY{c+c1}{\PYZsh{} Display sample data}
\PY{n}{df\PYZus{}market\PYZus{}data\PYZus{}scaled}\PY{o}{.}\PY{n}{head}\PY{p}{(}\PY{p}{)}
\end{Verbatim}
\end{tcolorbox}

    \begin{center}\rule{0.5\linewidth}{0.5pt}\end{center}

    \hypertarget{find-the-best-value-for-k-using-the-original-data}{%
\subsubsection{Find the Best Value for k Using the Original
Data}\label{find-the-best-value-for-k-using-the-original-data}}

In this section, you will use the elbow method to find the best value
for \texttt{k}.

\begin{enumerate}
\def\labelenumi{\arabic{enumi}.}
\item
  Code the elbow method algorithm to find the best value for \texttt{k}.
  Use a range from 1 to 11.
\item
  Plot a line chart with all the inertia values computed with the
  different values of \texttt{k} to visually identify the optimal value
  for \texttt{k}.
\item
  Answer the following question: What is the best value for \texttt{k}?
\end{enumerate}

    \begin{tcolorbox}[breakable, size=fbox, boxrule=1pt, pad at break*=1mm,colback=cellbackground, colframe=cellborder]
\prompt{In}{incolor}{ }{\boxspacing}
\begin{Verbatim}[commandchars=\\\{\}]
\PY{c+c1}{\PYZsh{} Create a list with the number of k\PYZhy{}values to try}
\PY{c+c1}{\PYZsh{} Use a range from 1 to 11}
\PY{c+c1}{\PYZsh{} YOUR CODE HERE!}
\end{Verbatim}
\end{tcolorbox}

    \begin{tcolorbox}[breakable, size=fbox, boxrule=1pt, pad at break*=1mm,colback=cellbackground, colframe=cellborder]
\prompt{In}{incolor}{ }{\boxspacing}
\begin{Verbatim}[commandchars=\\\{\}]
\PY{c+c1}{\PYZsh{} Create an empy list to store the inertia values}
\PY{c+c1}{\PYZsh{} YOUR CODE HERE!}
\end{Verbatim}
\end{tcolorbox}

    \begin{tcolorbox}[breakable, size=fbox, boxrule=1pt, pad at break*=1mm,colback=cellbackground, colframe=cellborder]
\prompt{In}{incolor}{ }{\boxspacing}
\begin{Verbatim}[commandchars=\\\{\}]
\PY{c+c1}{\PYZsh{} Create a for loop to compute the inertia with each possible value of k}
\PY{c+c1}{\PYZsh{} Inside the loop:}
\PY{c+c1}{\PYZsh{} 1. Create a KMeans model using the loop counter for the n\PYZus{}clusters}
\PY{c+c1}{\PYZsh{} 2. Fit the model to the data using `df\PYZus{}market\PYZus{}data\PYZus{}scaled`}
\PY{c+c1}{\PYZsh{} 3. Append the model.inertia\PYZus{} to the inertia list}
\PY{c+c1}{\PYZsh{} YOUR CODE HERE!}
\end{Verbatim}
\end{tcolorbox}

    \begin{tcolorbox}[breakable, size=fbox, boxrule=1pt, pad at break*=1mm,colback=cellbackground, colframe=cellborder]
\prompt{In}{incolor}{ }{\boxspacing}
\begin{Verbatim}[commandchars=\\\{\}]
\PY{c+c1}{\PYZsh{} Create a dictionary with the data to plot the Elbow curve}
\PY{c+c1}{\PYZsh{} YOUR CODE HERE!}

\PY{c+c1}{\PYZsh{} Create a DataFrame with the data to plot the Elbow curve}
\PY{c+c1}{\PYZsh{} YOUR CODE HERE!}
\end{Verbatim}
\end{tcolorbox}

    \begin{tcolorbox}[breakable, size=fbox, boxrule=1pt, pad at break*=1mm,colback=cellbackground, colframe=cellborder]
\prompt{In}{incolor}{ }{\boxspacing}
\begin{Verbatim}[commandchars=\\\{\}]
\PY{c+c1}{\PYZsh{} Plot a line chart with all the inertia values computed with }
\PY{c+c1}{\PYZsh{} the different values of k to visually identify the optimal value for k.}
\PY{c+c1}{\PYZsh{} YOUR CODE HERE!}
\end{Verbatim}
\end{tcolorbox}

    \hypertarget{answer-the-following-question-what-is-the-best-value-for-k}{%
\paragraph{Answer the following question: What is the best value for
k?}\label{answer-the-following-question-what-is-the-best-value-for-k}}

\textbf{Question:} What is the best value for \texttt{k}?

\textbf{Answer:} \# YOUR ANSWER HERE!

    \begin{center}\rule{0.5\linewidth}{0.5pt}\end{center}

    \hypertarget{cluster-cryptocurrencies-with-k-means-using-the-original-data}{%
\subsubsection{Cluster Cryptocurrencies with K-means Using the Original
Data}\label{cluster-cryptocurrencies-with-k-means-using-the-original-data}}

In this section, you will use the K-Means algorithm with the best value
for \texttt{k} found in the previous section to cluster the
cryptocurrencies according to the price changes of cryptocurrencies
provided.

\begin{enumerate}
\def\labelenumi{\arabic{enumi}.}
\item
  Initialize the K-Means model with four clusters using the best value
  for \texttt{k}.
\item
  Fit the K-Means model using the original data.
\item
  Predict the clusters to group the cryptocurrencies using the original
  data. View the resulting array of cluster values.
\item
  Create a copy of the original data and add a new column with the
  predicted clusters.
\item
  Create a scatter plot using hvPlot by setting
  \texttt{x="price\_change\_percentage\_24h"} and
  \texttt{y="price\_change\_percentage\_7d"}. Color the graph points
  with the labels found using K-Means and add the crypto name in the
  \texttt{hover\_cols} parameter to identify the cryptocurrency
  represented by each data point.
\end{enumerate}

    \begin{tcolorbox}[breakable, size=fbox, boxrule=1pt, pad at break*=1mm,colback=cellbackground, colframe=cellborder]
\prompt{In}{incolor}{ }{\boxspacing}
\begin{Verbatim}[commandchars=\\\{\}]
\PY{c+c1}{\PYZsh{} Initialize the K\PYZhy{}Means model using the best value for k}
\PY{c+c1}{\PYZsh{} YOUR CODE HERE!}
\end{Verbatim}
\end{tcolorbox}

    \begin{tcolorbox}[breakable, size=fbox, boxrule=1pt, pad at break*=1mm,colback=cellbackground, colframe=cellborder]
\prompt{In}{incolor}{ }{\boxspacing}
\begin{Verbatim}[commandchars=\\\{\}]
\PY{c+c1}{\PYZsh{} Fit the K\PYZhy{}Means model using the scaled data}
\PY{c+c1}{\PYZsh{} YOUR CODE HERE!}
\end{Verbatim}
\end{tcolorbox}

    \begin{tcolorbox}[breakable, size=fbox, boxrule=1pt, pad at break*=1mm,colback=cellbackground, colframe=cellborder]
\prompt{In}{incolor}{ }{\boxspacing}
\begin{Verbatim}[commandchars=\\\{\}]
\PY{c+c1}{\PYZsh{} Predict the clusters to group the cryptocurrencies using the scaled data}
\PY{c+c1}{\PYZsh{} YOUR CODE HERE!}

\PY{c+c1}{\PYZsh{} View the resulting array of cluster values.}
\PY{c+c1}{\PYZsh{} YOUR CODE HERE!}
\end{Verbatim}
\end{tcolorbox}

    \begin{tcolorbox}[breakable, size=fbox, boxrule=1pt, pad at break*=1mm,colback=cellbackground, colframe=cellborder]
\prompt{In}{incolor}{ }{\boxspacing}
\begin{Verbatim}[commandchars=\\\{\}]
\PY{c+c1}{\PYZsh{} Create a copy of the DataFrame}
\PY{c+c1}{\PYZsh{} YOUR CODE HERE!}
\end{Verbatim}
\end{tcolorbox}

    \begin{tcolorbox}[breakable, size=fbox, boxrule=1pt, pad at break*=1mm,colback=cellbackground, colframe=cellborder]
\prompt{In}{incolor}{ }{\boxspacing}
\begin{Verbatim}[commandchars=\\\{\}]
\PY{c+c1}{\PYZsh{} Add a new column to the DataFrame with the predicted clusters}
\PY{c+c1}{\PYZsh{} YOUR CODE HERE!}

\PY{c+c1}{\PYZsh{} Display sample data}
\PY{c+c1}{\PYZsh{} YOUR CODE HERE!}
\end{Verbatim}
\end{tcolorbox}

    \begin{tcolorbox}[breakable, size=fbox, boxrule=1pt, pad at break*=1mm,colback=cellbackground, colframe=cellborder]
\prompt{In}{incolor}{ }{\boxspacing}
\begin{Verbatim}[commandchars=\\\{\}]
\PY{c+c1}{\PYZsh{} Create a scatter plot using hvPlot by setting }
\PY{c+c1}{\PYZsh{} `x=\PYZdq{}price\PYZus{}change\PYZus{}percentage\PYZus{}24h\PYZdq{}` and `y=\PYZdq{}price\PYZus{}change\PYZus{}percentage\PYZus{}7d\PYZdq{}`. }
\PY{c+c1}{\PYZsh{} Color the graph points with the labels found using K\PYZhy{}Means and }
\PY{c+c1}{\PYZsh{} add the crypto name in the `hover\PYZus{}cols` parameter to identify }
\PY{c+c1}{\PYZsh{} the cryptocurrency represented by each data point.}
\PY{c+c1}{\PYZsh{} YOUR CODE HERE!}
\end{Verbatim}
\end{tcolorbox}

    \begin{center}\rule{0.5\linewidth}{0.5pt}\end{center}

    \hypertarget{optimize-clusters-with-principal-component-analysis}{%
\subsubsection{Optimize Clusters with Principal Component
Analysis}\label{optimize-clusters-with-principal-component-analysis}}

In this section, you will perform a principal component analysis (PCA)
and reduce the features to three principal components.

\begin{enumerate}
\def\labelenumi{\arabic{enumi}.}
\item
  Create a PCA model instance and set \texttt{n\_components=3}.
\item
  Use the PCA model to reduce to three principal components. View the
  first five rows of the DataFrame.
\item
  Retrieve the explained variance to determine how much information can
  be attributed to each principal component.
\item
  Answer the following question: What is the total explained variance of
  the three principal components?
\item
  Create a new DataFrame with the PCA data. Be sure to set the
  \texttt{coin\_id} index from the original DataFrame as the index for
  the new DataFrame. Review the resulting DataFrame.
\end{enumerate}

    \begin{tcolorbox}[breakable, size=fbox, boxrule=1pt, pad at break*=1mm,colback=cellbackground, colframe=cellborder]
\prompt{In}{incolor}{ }{\boxspacing}
\begin{Verbatim}[commandchars=\\\{\}]
\PY{c+c1}{\PYZsh{} Create a PCA model instance and set `n\PYZus{}components=3`.}
\PY{c+c1}{\PYZsh{} YOUR CODE HERE!}
\end{Verbatim}
\end{tcolorbox}

    \begin{tcolorbox}[breakable, size=fbox, boxrule=1pt, pad at break*=1mm,colback=cellbackground, colframe=cellborder]
\prompt{In}{incolor}{ }{\boxspacing}
\begin{Verbatim}[commandchars=\\\{\}]
\PY{c+c1}{\PYZsh{} Use the PCA model with `fit\PYZus{}transform` to reduce to }
\PY{c+c1}{\PYZsh{} three principal components.}
\PY{c+c1}{\PYZsh{} YOUR CODE HERE!}

\PY{c+c1}{\PYZsh{} View the first five rows of the DataFrame. }
\PY{c+c1}{\PYZsh{} YOUR CODE HERE!}
\end{Verbatim}
\end{tcolorbox}

    \begin{tcolorbox}[breakable, size=fbox, boxrule=1pt, pad at break*=1mm,colback=cellbackground, colframe=cellborder]
\prompt{In}{incolor}{ }{\boxspacing}
\begin{Verbatim}[commandchars=\\\{\}]
\PY{c+c1}{\PYZsh{} Retrieve the explained variance to determine how much information }
\PY{c+c1}{\PYZsh{} can be attributed to each principal component.}
\PY{c+c1}{\PYZsh{} YOUR CODE HERE!}
\end{Verbatim}
\end{tcolorbox}

    \hypertarget{answer-the-following-question-what-is-the-total-explained-variance-of-the-three-principal-components}{%
\paragraph{Answer the following question: What is the total explained
variance of the three principal
components?}\label{answer-the-following-question-what-is-the-total-explained-variance-of-the-three-principal-components}}

\textbf{Question:} What is the total explained variance of the three
principal components?

\textbf{Answer:} \# YOUR ANSWER HERE!

    \begin{tcolorbox}[breakable, size=fbox, boxrule=1pt, pad at break*=1mm,colback=cellbackground, colframe=cellborder]
\prompt{In}{incolor}{ }{\boxspacing}
\begin{Verbatim}[commandchars=\\\{\}]
\PY{c+c1}{\PYZsh{} Create a new DataFrame with the PCA data.}
\PY{c+c1}{\PYZsh{} Note: The code for this step is provided for you}

\PY{c+c1}{\PYZsh{} Creating a DataFrame with the PCA data}
\PY{c+c1}{\PYZsh{} YOUR CODE HERE!}

\PY{c+c1}{\PYZsh{} Copy the crypto names from the original data}
\PY{c+c1}{\PYZsh{} YOUR CODE HERE!}

\PY{c+c1}{\PYZsh{} Set the coinid column as index}
\PY{c+c1}{\PYZsh{} YOUR CODE HERE!}

\PY{c+c1}{\PYZsh{} Display sample data}
\PY{c+c1}{\PYZsh{} YOUR CODE HERE!}
\end{Verbatim}
\end{tcolorbox}

    \begin{center}\rule{0.5\linewidth}{0.5pt}\end{center}

    \hypertarget{find-the-best-value-for-k-using-the-pca-data}{%
\subsubsection{Find the Best Value for k Using the PCA
Data}\label{find-the-best-value-for-k-using-the-pca-data}}

In this section, you will use the elbow method to find the best value
for \texttt{k} using the PCA data.

\begin{enumerate}
\def\labelenumi{\arabic{enumi}.}
\item
  Code the elbow method algorithm and use the PCA data to find the best
  value for \texttt{k}. Use a range from 1 to 11.
\item
  Plot a line chart with all the inertia values computed with the
  different values of \texttt{k} to visually identify the optimal value
  for \texttt{k}.
\item
  Answer the following questions: What is the best value for k when
  using the PCA data? Does it differ from the best k value found using
  the original data?
\end{enumerate}

    \begin{tcolorbox}[breakable, size=fbox, boxrule=1pt, pad at break*=1mm,colback=cellbackground, colframe=cellborder]
\prompt{In}{incolor}{ }{\boxspacing}
\begin{Verbatim}[commandchars=\\\{\}]
\PY{c+c1}{\PYZsh{} Create a list with the number of k\PYZhy{}values to try}
\PY{c+c1}{\PYZsh{} Use a range from 1 to 11}
\PY{c+c1}{\PYZsh{} YOUR CODE HERE!}
\end{Verbatim}
\end{tcolorbox}

    \begin{tcolorbox}[breakable, size=fbox, boxrule=1pt, pad at break*=1mm,colback=cellbackground, colframe=cellborder]
\prompt{In}{incolor}{ }{\boxspacing}
\begin{Verbatim}[commandchars=\\\{\}]
\PY{c+c1}{\PYZsh{} Create an empy list to store the inertia values}
\PY{c+c1}{\PYZsh{} YOUR CODE HERE!}
\end{Verbatim}
\end{tcolorbox}

    \begin{tcolorbox}[breakable, size=fbox, boxrule=1pt, pad at break*=1mm,colback=cellbackground, colframe=cellborder]
\prompt{In}{incolor}{ }{\boxspacing}
\begin{Verbatim}[commandchars=\\\{\}]
\PY{c+c1}{\PYZsh{} Create a for loop to compute the inertia with each possible value of k}
\PY{c+c1}{\PYZsh{} Inside the loop:}
\PY{c+c1}{\PYZsh{} 1. Create a KMeans model using the loop counter for the n\PYZus{}clusters}
\PY{c+c1}{\PYZsh{} 2. Fit the model to the data using `df\PYZus{}market\PYZus{}data\PYZus{}pca`}
\PY{c+c1}{\PYZsh{} 3. Append the model.inertia\PYZus{} to the inertia list}
\PY{c+c1}{\PYZsh{} YOUR CODE HERE!}
\end{Verbatim}
\end{tcolorbox}

    \begin{tcolorbox}[breakable, size=fbox, boxrule=1pt, pad at break*=1mm,colback=cellbackground, colframe=cellborder]
\prompt{In}{incolor}{ }{\boxspacing}
\begin{Verbatim}[commandchars=\\\{\}]
\PY{c+c1}{\PYZsh{} Create a dictionary with the data to plot the Elbow curve}
\PY{c+c1}{\PYZsh{} YOUR CODE HERE!}

\PY{c+c1}{\PYZsh{} Create a DataFrame with the data to plot the Elbow curve}
\PY{c+c1}{\PYZsh{} YOUR CODE HERE!}
\end{Verbatim}
\end{tcolorbox}

    \begin{tcolorbox}[breakable, size=fbox, boxrule=1pt, pad at break*=1mm,colback=cellbackground, colframe=cellborder]
\prompt{In}{incolor}{ }{\boxspacing}
\begin{Verbatim}[commandchars=\\\{\}]
\PY{c+c1}{\PYZsh{} Plot a line chart with all the inertia values computed with }
\PY{c+c1}{\PYZsh{} the different values of k to visually identify the optimal value for k.}
\PY{c+c1}{\PYZsh{} YOUR CODE HERE!}
\end{Verbatim}
\end{tcolorbox}

    \hypertarget{answer-the-following-questions-what-is-the-best-value-for-k-when-using-the-pca-data-does-it-differ-from-the-best-k-value-found-using-the-original-data}{%
\paragraph{Answer the following questions: What is the best value for k
when using the PCA data? Does it differ from the best k value found
using the original
data?}\label{answer-the-following-questions-what-is-the-best-value-for-k-when-using-the-pca-data-does-it-differ-from-the-best-k-value-found-using-the-original-data}}

\begin{itemize}
\item
  \textbf{Question:} What is the best value for \texttt{k} when using
  the PCA data?

  \begin{itemize}
  \tightlist
  \item
    \textbf{Answer:} \# YOUR ANSWER HERE!
  \end{itemize}
\item
  \textbf{Question:} Does it differ from the best k value found using
  the original data?

  \begin{itemize}
  \tightlist
  \item
    \textbf{Answer:} \# YOUR ANSWER HERE!
  \end{itemize}
\end{itemize}

    \begin{center}\rule{0.5\linewidth}{0.5pt}\end{center}

    \hypertarget{cluster-cryptocurrencies-with-k-means-using-the-pca-data}{%
\subsubsection{Cluster Cryptocurrencies with K-means Using the PCA
Data}\label{cluster-cryptocurrencies-with-k-means-using-the-pca-data}}

In this section, you will use the PCA data and the K-Means algorithm
with the best value for \texttt{k} found in the previous section to
cluster the cryptocurrencies according to the principal components.

\begin{enumerate}
\def\labelenumi{\arabic{enumi}.}
\item
  Initialize the K-Means model with four clusters using the best value
  for \texttt{k}.
\item
  Fit the K-Means model using the PCA data.
\item
  Predict the clusters to group the cryptocurrencies using the PCA data.
  View the resulting array of cluster values.
\item
  Add a new column to the DataFrame with the PCA data to store the
  predicted clusters.
\item
  Create a scatter plot using hvPlot by setting \texttt{x="PC1"} and
  \texttt{y="PC2"}. Color the graph points with the labels found using
  K-Means and add the crypto name in the \texttt{hover\_cols} parameter
  to identify the cryptocurrency represented by each data point.
\end{enumerate}

    \begin{tcolorbox}[breakable, size=fbox, boxrule=1pt, pad at break*=1mm,colback=cellbackground, colframe=cellborder]
\prompt{In}{incolor}{ }{\boxspacing}
\begin{Verbatim}[commandchars=\\\{\}]
\PY{c+c1}{\PYZsh{} Initialize the K\PYZhy{}Means model using the best value for k}
\PY{c+c1}{\PYZsh{} YOUR CODE HERE!}
\end{Verbatim}
\end{tcolorbox}

    \begin{tcolorbox}[breakable, size=fbox, boxrule=1pt, pad at break*=1mm,colback=cellbackground, colframe=cellborder]
\prompt{In}{incolor}{ }{\boxspacing}
\begin{Verbatim}[commandchars=\\\{\}]
\PY{c+c1}{\PYZsh{} Fit the K\PYZhy{}Means model using the PCA data}
\PY{c+c1}{\PYZsh{} YOUR CODE HERE!}
\end{Verbatim}
\end{tcolorbox}

    \begin{tcolorbox}[breakable, size=fbox, boxrule=1pt, pad at break*=1mm,colback=cellbackground, colframe=cellborder]
\prompt{In}{incolor}{ }{\boxspacing}
\begin{Verbatim}[commandchars=\\\{\}]
\PY{c+c1}{\PYZsh{} Predict the clusters to group the cryptocurrencies using the PCA data}
\PY{c+c1}{\PYZsh{} YOUR CODE HERE!}

\PY{c+c1}{\PYZsh{} View the resulting array of cluster values.}
\PY{c+c1}{\PYZsh{} YOUR CODE HERE!}
\end{Verbatim}
\end{tcolorbox}

    \begin{tcolorbox}[breakable, size=fbox, boxrule=1pt, pad at break*=1mm,colback=cellbackground, colframe=cellborder]
\prompt{In}{incolor}{ }{\boxspacing}
\begin{Verbatim}[commandchars=\\\{\}]
\PY{c+c1}{\PYZsh{} Create a copy of the DataFrame with the PCA data}
\PY{c+c1}{\PYZsh{} YOUR CODE HERE!}

\PY{c+c1}{\PYZsh{} Add a new column to the DataFrame with the predicted clusters}
\PY{c+c1}{\PYZsh{} YOUR CODE HERE!}

\PY{c+c1}{\PYZsh{} Display sample data}
\PY{c+c1}{\PYZsh{} YOUR CODE HERE!}
\end{Verbatim}
\end{tcolorbox}

    \begin{tcolorbox}[breakable, size=fbox, boxrule=1pt, pad at break*=1mm,colback=cellbackground, colframe=cellborder]
\prompt{In}{incolor}{ }{\boxspacing}
\begin{Verbatim}[commandchars=\\\{\}]
\PY{c+c1}{\PYZsh{} Create a scatter plot using hvPlot by setting }
\PY{c+c1}{\PYZsh{} `x=\PYZdq{}PC1\PYZdq{}` and `y=\PYZdq{}PC2\PYZdq{}`. }
\PY{c+c1}{\PYZsh{} Color the graph points with the labels found using K\PYZhy{}Means and }
\PY{c+c1}{\PYZsh{} add the crypto name in the `hover\PYZus{}cols` parameter to identify }
\PY{c+c1}{\PYZsh{} the cryptocurrency represented by each data point.}
\PY{c+c1}{\PYZsh{} YOUR CODE HERE!}
\end{Verbatim}
\end{tcolorbox}

    \begin{center}\rule{0.5\linewidth}{0.5pt}\end{center}

    \hypertarget{visualize-and-compare-the-results}{%
\subsubsection{Visualize and Compare the
Results}\label{visualize-and-compare-the-results}}

In this section, you will visually analyze the cluster analysis results
by contrasting the outcome with and without using the optimization
techniques.

\begin{enumerate}
\def\labelenumi{\arabic{enumi}.}
\item
  Create a composite plot using hvPlot and the plus (\texttt{+})
  operator to contrast the Elbow Curve that you created to find the best
  value for \texttt{k} with the original and the PCA data.
\item
  Create a composite plot using hvPlot and the plus (\texttt{+})
  operator to contrast the cryptocurrencies clusters using the original
  and the PCA data.
\item
  Answer the following question: After visually analyzing the cluster
  analysis results, what is the impact of using fewer features to
  cluster the data using K-Means?
\end{enumerate}

\begin{quote}
\textbf{Rewind:} Back in Lesson 3 of Module 6, you learned how to create
composite plots. You can look at that lesson to review how to make these
plots; also, you can check
\href{https://holoviz.org/tutorial/Composing_Plots.html}{the hvPlot
documentation}.
\end{quote}

    \begin{tcolorbox}[breakable, size=fbox, boxrule=1pt, pad at break*=1mm,colback=cellbackground, colframe=cellborder]
\prompt{In}{incolor}{ }{\boxspacing}
\begin{Verbatim}[commandchars=\\\{\}]
\PY{c+c1}{\PYZsh{} Composite plot to contrast the Elbow curves}
\PY{c+c1}{\PYZsh{} YOUR CODE HERE!}
\end{Verbatim}
\end{tcolorbox}

    \begin{tcolorbox}[breakable, size=fbox, boxrule=1pt, pad at break*=1mm,colback=cellbackground, colframe=cellborder]
\prompt{In}{incolor}{ }{\boxspacing}
\begin{Verbatim}[commandchars=\\\{\}]
\PY{c+c1}{\PYZsh{} Compoosite plot to contrast the clusters}
\PY{c+c1}{\PYZsh{} YOUR CODE HERE!}
\end{Verbatim}
\end{tcolorbox}

    \hypertarget{answer-the-following-question-after-visually-analyzing-the-cluster-analysis-results-what-is-the-impact-of-using-fewer-features-to-cluster-the-data-using-k-means}{%
\paragraph{Answer the following question: After visually analyzing the
cluster analysis results, what is the impact of using fewer features to
cluster the data using
K-Means?}\label{answer-the-following-question-after-visually-analyzing-the-cluster-analysis-results-what-is-the-impact-of-using-fewer-features-to-cluster-the-data-using-k-means}}

\begin{itemize}
\item
  \textbf{Question:} After visually analyzing the cluster analysis
  results, what is the impact of using fewer features to cluster the
  data using K-Means?
\item
  \textbf{Answer:} \# YOUR ANSWER HERE!
\end{itemize}


    % Add a bibliography block to the postdoc
    
    
    
\end{document}
